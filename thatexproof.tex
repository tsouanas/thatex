%{{{ [vim] 
% vim:foldmarker=%{{{,%}}}
% vim:foldmethod=marker
% vim:foldcolumn=4
%}}}
%% thatexproof.tex
%% author: Thanos Tsouanas <thanos@tsouanas.org>

% You must first load bussproofs.sty and string.sty
% If used outside of thatex, you must define
% \let\IfStrContains=\IfSubStr\relax
% \def\WhenStrContains#1#2#3{\IfStrContains {#1} {#2} {#3} {}}

% TODO: \I's ##2 argument is ignored.
% It should be used to determine the style of inference line: --, ==, etc.

\def\Assumption#1{\AxiomC{#1}}
\def\InfFromBy#1#2#3#4{% #1: arity; #2: style; #3: label; #4: conclusion
% write \RightLabel{#3} only for non-0 case
\ifcase #1
\AxiomC{}
\fi
% write \RightLabel{#3} in all cases
\RightLabel{#3}
% set style
\WhenStrContains {#2} {*****} {\noLine}                % no line
\WhenStrContains {#2} {-----} {\singleLine\solidLine}  % single solid
\WhenStrContains {#2} {=====} {\doubleLine\solidLine}  % double solid
\WhenStrContains {#2} {.....} {\singleLine\dottedLine} % single dotted
\WhenStrContains {#2} {:::::} {\doubleLine\dottedLine} % double dotted
\WhenStrContains {#2} {'''''} {\singleLine\dashedLine} % single dashed
\WhenStrContains {#2} {"""""} {\doubleLine\dashedLine} % double dashed
% infer conclusion
\ifcase #1 \UnaryInfC{#4}       % 0
\or        \UnaryInfC{#4}       % 1
\or        \BinaryInfC{#4}      % 2
\or        \TrinaryInfC{#4}     % 3
\or        \QuaternaryInfC{#4}  % 4
\or        \QuinaryInfC{#4}     % 5
\fi
}

% PROOF{,m,r,mn,mr,mrw}:
% m = assume math mode for nodes
% r = put rule labels in \rulelabel{...}
% w = space-delimited args (words)
% n = naked (no labels)

\def\PROOF#1{
\def\A{\Assumption}
\def\I##1##2 ##3##4{\InfFromBy##1{##2}{##3}{##4}}
#1
\DisplayProof
}

\def\PROOFm#1{
\def\A##1{\Assumption{$##1$}}
\def\I##1##2 ##3##4{\InfFromBy##1{##2}{##3}{$##4$}}
#1
\DisplayProof
}

\def\PROOFmn#1{
\def\A##1{\Assumption{$##1$}}
\def\I##1##2 ##3{\InfFromBy##1{##2}{}{$##3$}}
#1
\DisplayProof
}

\def\PROOFr#1{
\def\A{\Assumption}
\def\I##1##2 ##3##4{\InfFromBy##1{##2}{\rulelabel{##3}}{##4}}
#1
\DisplayProof
}

\def\PROOFmr#1{
\def\A##1{\Assumption{$##1$}}
\def\I##1##2 ##3##4{\InfFromBy##1{##2}{\rulelabel{##3}}{$##4$}}
#1
\DisplayProof
}

\def\PROOFmrw#1{
\def\A##1 {\Assumption{$##1$}}
\def\I##1##2 ##3 ##4 {\InfFromBy##1{##2}{\rulelabel{##3}}{$##4$}}
#1
\DisplayProof
}

\def\infered#1\endinfered{\PROOF{#1}}
\def\infer#1\endinfer{\math#1\endmath}

