%{{{ [vim] 
% vim:foldmarker=%{{{,%}}}
% vim:foldmethod=marker
% vim:foldcolumn=4
%}}}
%% thatex.tex
%% author: Thanos Tsouanas <thanos@tsouanas.org>

\message{[ThaTeX]}

%%{{{ \inputs 

% xstring <3
\input xstring

% OPmac
% XXX: weird conflict with eplain-based labs/refs if loaded after eplain
%\input thatex-opmac
%\hyperlinks \Red \Blue

% eplain
\let\noarrow=t
\input thatex-eplain
\let\noarrow\relax
\def\xrefpageword{}

% hiding:
%\let\eplainref=\ref % (\eplainref is already defined by eplain)
\let\eplainrefn=\refn
\let\eplainrefs=\refs
\let\eplaineqref=\eqref
\let\eplaineqrefn=\eqrefn
\let\eplaineqdef=\eqdef
\let\eplaineqdefn=\eqdefn
\let\eplaincite=\cite
\let\ref=\undefined
\let\refs=\undefined
\let\refn=\undefined
\let\eqref=\undefined
\let\eqrefn=\undefined
\let\eqdef=\undefined
\let\cite=\undefined

% these must be loaded before the following LaTeX packages
\input bussproofs.sty
\input xkeyval

% LaTeX packages
\beginpackages
\usepackage{url}
\usepackage[dvipsnames]{color}
\usepackage{graphicx}
\endpackages

% protect colors from tiks/pgfmath 
\let\eplaincolor\color
\let\eplaindefinecolor\definecolor
\input tikz % includes pgffor and pgfkeys
\input pgfplots
\usetikzlibrary{shapes}
\let\color\eplaincolor
\let\definecolor\eplaindefinecolor

% thatex-repeat has \Repeat so it's safe
\input thatex-repeat

% protect other possibly overwritten macros
\let\knuthll=\ll
\let\knuthast=\ast
\let\knuthcirc=\circ
\let\knuthbullet=\bullet
\let\knuthcdot=\cdot
\let\knuthS=\S
\let\knuthP=\P
\let\knuthTILDE=~

% not currently used
%\input insbox

%%}}}

%%{{{ ThaTeX hyperlinks 
% Hyperlinks woohoo!
\enablehyperlinks [dvipdfm]
%\definecolor{urlcolor}{rgb}{.7,.2,.2}
\hlopts{bwidth=0}
\hlopts{colormodel=cmyk,color={.4,1,.6,.1}}
\hlopts[url]{colormodel=cmyk,color={1,.4,.6,.1}}
\hlopts[hrefext]{colormodel=cmyk,color={1,.4,.6,.1}}
\hlopts[eq]{colormodel=cmyk,color={1,1,0,.1}}
\def\UrlFont{\twelvett}
%\hlopts[page]{colormodel=cmyk,color={1,.2,.1,.5}}
\def\hlprintpage#1{\hlstart{name}{}{TOCPG#1}#1\hlend}
\def\hldestizepage{\hldest{fit}{}{TOCPG\the\pageno}}
%%}}}

%%{{{ ThaTeX sidekick 

\def\ThaTeX{\hbox{$\Theta$\kern-.1333em\lower.5ex\hbox{${\acute\alpha}$}\kern-.21em\TeX}}
\def\thatexmsg[#1]{\message{[ThaTeX]: #1}}
\def\thatexwarn[#1]{\thatexmsg[Warning: #1]}
\def\thatexerror[#1]{\thatexmsg[******** ERROR: #1]\ThaTeXERROR}
\long\def\thatexdebug[#1]{\thatexmsg[****************************************************************]\message{#1}\thatexmsg[****************************************************************]}
\let\ea=\expandafter

% xstring love
\def\IfStrEmpty#1#2#3{\IfStrEq {#1} {} {#2} {#3}}
\def\IfStrNempty#1#2#3{\IfStrEq {#1} {} {#3} {#2}}
% IfSubStr has a bad name
\let\IfStrContains=\IfSubStr\relax
\def\IfSubStr#1#2#3{\IfStrContains {#2} {#1} {#3}}
\def\WhenStrEq#1#2#3{\IfStrEq {#1} {#2} {#3} {}}
\def\WhenStrNeq#1#2#3{\IfStrEq {#1} {#2} {} {#3}}
\def\WhenStrEmpty#1#2{\IfStrEq {#1} {} {#2} {}}
\def\WhenStrNempty#1#2{\IfStrEq {#1} {} {} {#2}}
\def\WhenStrContains#1#2#3{\IfStrContains {#1} {#2} {#3} {}}
\def\WhenStrNcontains#1#2#3{\IfStrContains {#1} {#2} {} {#3}}
\let\IfStrBeginsWith=\IfBeginWith
\let\IfStrEndsWith=\IfEndWith

% letters 
\def\letterm{{m}}%
\def\lettert{{t}}%
\def\letterc{{c}}%

% used all over the place in \ifx\NIL..\fi patterns
\def\NIL{}

\def\ifblank#1{\ifx\empty#1}
\def\ifvanishes#1\\{\csgroup\ifx{empty#1}\empty}
% XXX: ifnil has a bad name because of \NIL defined above
\def\ifnil#1\\{\if\relax\detokenize{#1}\relax}% needs e-TeX
\def\twodigitfmt#1{\ifnum #1<10 0\fi#1}
\def\twopadthe#1{\ifnum #1<10 \phantom0\fi\the#1}

% taken from opmac for when we're not \input ing it
\def\eoldef#1{\def#1{\begingroup \catcode`\^^M=12 \eoldefA#1}%
   \ea\def\csname\string#1:M\endcsname}
{\catcode`\^^M=12 \gdef\eoldefA#1#2^^M{\endgroup\csname\string#1:M\endcsname{#2}}}
\long\def\addto#1#2{\ea\def\ea#1\ea{#1#2}}
\long\def\gaddto#1#2{\ea\gdef\ea#1\ea{#1#2}} % not in opmac
\def\adef#1{\catcode`#1=13 \begingroup \lccode`\~=`#1\lowercase{\endgroup\def~}}

\long\def\stripgroup#1{#1}% XXX/TODO
\def\eatfst#1{}
\def\eatsnd#1#2{#1}
% fullyexpanded: \romannumeral-`0...
\def\fullyexpand{\romannumeral - `0}

% check Ulrich Diez's answer at:
% https://tex.stackexchange.com/questions/543592/using-let-with-csname-inside-a-macro/543600
% csgroup \csname'izes the next {..} in sight (even if far: \long)
% you can use multiple, e.g.:
% \csgroup\csgroup....{fst}....{snd} -> \csgroup...\fst...{snd} -> ...\fst...\snd
\long\def\csgroup#1#{\fullyexpand\SIDEKICKcsgroup{#1}}
% made this \long
\long\def\SIDEKICKcsgroup#1#2{\ea\swapcs\ea{\csname#2\endcsname}{ #1}}%
% made this \long
\long\def\swapcs#1#2{#2#1}

%% \safelet mac omac : lets mac=omac unless mac is already set
\def\safelet#1 #2 {\ifcsname #1\endcsname\else\csgroup\csgroup\let{#1}={#2}}

%% Mixin parameters if available
\def\safeset#1#2#3{\ifcsname #2#3\endcsname\csgroup\csgroup\let{#1#3}={#2#3}\fi}
\def\Mixin #1\from #2 \to #3 {%
\SIDEKICKDefSafeSetter{#3}{#2}%
\csgroup{SafeSet#3}#1 ;
\SIDEKICKUndefSafeSetter{#3}%
}
\def\SIDEKICKDefSafeSetter#1#2{%
\csgroup\def{safeset#1}##1{\safeset{#1}{#2}{##1}}%
\csgroup\def{SafeSet#1}##1 ; {\csgroup\Maps{safeset#1}{##1}}%
}
\def\SIDEKICKUndefSafeSetter#1{\csgroup\undef{safeset#1}\csgroup\undef{SafeSet#1}}

\def\CAR#1#2\NIL{#1}
\def\CDR#1#2\NIL{#2}

\def\spacechar{ }

% \DefStyle foo {\bf} <- this will define \foostyle (nullary) and \foostylize (unary)
\def\DefStyle #1 #{\csgroup\def{#1stylize}##1{{\csgroup{#1style}##1}}\csgroup\def{#1style}}

\protected\def\safedef#1#2#{\ifx\relax#1\else\ifdefined#1\thatexerror[SafedefError: (\noexpand#1)]\fi\fi\def#1#2}
\protected\def\undef#1{\let#1=\undefined}
\def\undef#1{\let#1=\undefined}

\long\def\ignore#1{}
\def\nrepeat#1 #2{\Repeat \for{i} \from{1} \to{#1} \do{#2}}

% replace (without any expansion)
\def\replace[#1=#2]#3\\{% by David Carlisle
\def\stopreplace##1\stopreplace{}
\def\delimreplacer##1#1{##1#2\delimreplacer}%
\delimreplacer#3\stopreplace#1\stopreplace}

% same thing but different order
% (useful for \ea in \Replace below)
\def\replacein#1[#2=#3]{\replace[#2=#3]#1\\}

% expand and replace
\def\Replace[#1=#2]#3\\{% deep
\edef\ReplaceString{#3}%
\ea\replacein \ReplaceString[#1=#2]}

\def\Identity#1{#1}
\let\Id=\Identity

% composition of functions
\protected\def\defcompose#1#2#3{\def#1##1{#2{#3{##1}}}}
\def\Compose[#1 #2]{% XXX: experimental
\ea\def\csname COMPOSITION#1FOLLOWING#2\endcsname##1{\csname #1\endcsname{\csname #2\endcsname{##1}}}%
\csname COMPOSITION#1FOLLOWING#2\endcsname}
% TODO: make an n-ary Compose

\protected\def\Map#1#2#3{% [The Advanced TeXbook, 5.16]
% #1: separator
% #2: function
% #3: #1-separated list
\def\mapfun##1#1{%
\if;##1\let\next=\relax
\else\let\next=\mapfun
#2{##1}%
\fi\next}%
\mapfun #3#1;#1\let\mapfun=\relax
}

\protected\def\Maps#1#2{\Map{ }{#1}{#2}}

\def\toupper#1{\uppercase\ea{#1}}
\def\tolower#1{\lowercase\ea{#1}}

\def\qqquad{\hskip 3em\relax}
\def\qqqquad{\hskip 4em\relax}

% XXX: for use with qed symbols, etc.  Works well but not great.
\def\XXXeolllap#1{\unskip\nolinebreak\phantom{X}\nolinebreak\hfill\llap{#1}}
\def\XXXeolrlap#1{\unskip\nolinebreak\phantom{X}\nolinebreak\hfill\rlap{#1}}
% \eolfill ensures the next thing is placed at the end of the same or the next line
\def\eolfill{\ifhmode\unskip\nobreak\fi\hbox{}\hfill\hbox{}\nobreak}% XXX 
\def\XXXeolfill{\ifhmode\phantom.\fi\ifhmode\unskip\nobreak\fi\hbox{}\hfill\hbox{}\nobreak}% XXX 
\def\CR{\newline}
\def\boxit#1#2{\hbox{\vrule\vtop{\vbox{\hrule\kern#1\hbox{\kern#1#2\kern#1}}\kern#1\hrule}\vrule\relax}}
\def\frameit#1#2#3{\hbox{\vrule width #1\vtop{\vbox{\hrule height #1\kern#2\hbox{\kern#2#3\kern#2}}\kern#2\hrule height #1}\vrule width #2\relax}}
\let\footnote=\numberedfootnote
\def\foot#1\toof{\footnote{#1} }% XXX: ugly
\def\colorboxfb#1#2#3{\colorbox{#2}{\color{#1}#3}}% colorbox fg in bg
\def\procrustes{{\vphantom{Tj}}}
\def\procrustext#1{\text{\procrustes#1}}
\def\procruster{{\ensuremath{\vphantom{{}^T_j}}}}% even more extreme for math
\def\procrustest{{\ensuremath{\vphantom{{}^{T^T}_{j_j}}}}}% even more!

%%}}}

%%{{{ ThaTeX skips 

% N.B.: the default skip namings are inconsistent:
%       the amount of a baselineskip is called baselineskip

% meta: creates skip amount and corresponding skip command
\def\Newskip#1=#2; {%
\csgroup\newskip{#1skipamount}%
\csgroup{#1skipamount}=#2\relax
\csgroup\def{#1skip}{\csgroup\vskip{#1skipamount}\relax}%
\csgroup\def{#1glue}{\csgroup\vglue{#1skipamount}\relax}%
}

\Newskip tiny=2pt plus 1pt minus 1pt;

%%}}}

%%{{{ ThaTeX numerals 
\def\asciinumeral#1 offset #2{{\innernewcount\tmp\tmp=#1\relax\advance\tmp by #2\relax\char\tmp}}
\def\Alphabetnumeral#1{\asciinumeral#1 offset {64}}
\def\alphabetnumeral#1{\asciinumeral#1 offset {96}}

\def\nthe#1{\number#1}
\def\ithe#1{\romannumeral#1}
\def\Ithe#1{\uppercase\ea{\romannumeral#1}}
\def\athe#1{\ifcase#1?\or a\or b\or c\or d\or e\or f\or g\or h\or i\or j\or k\or l\or m\or n\or o\or p\or q\or r\or s\or t\or u\or v\or w\or x\or y\or z\else\thatexerror[atheOutOfRange]\fi}
\def\Athe#1{\ifcase#1?\or A\or B\or C\or D\or E\or F\or G\or H\or I\or J\or K\or L\or M\or N\or O\or P\or Q\or R\or S\or T\or U\or V\or W\or X\or Y\or Z\else\thatexerror[AtheOutOfRange]\fi}
\def\gthe#1{\ifcase#1?\or α\or β\or γ\or δ\or ε\or ς\or ζ\or η\or θ\or ι\or ια\or ιβ\or ιγ\or ιδ\or ιε\or ιϛ\or ιζ\or ιη\or ιθ\or κ\or κα\or κβ\or κγ\or κδ\or κε\or κς\or κζ\or κη\or κθ\else\thatexerror[alphatheOutOfRange]\fi ʹ}
\def\Gthe#1{\ifcase#1?\or Α\or Β\or Γ\or Δ\or Ε\or Σ\or Ζ\or Η\or Θ\or Ι\or ΙΑ\or ΙΒ\or ΙΓ\or ΙΔ\or ΙΕ\or ΙϚ\or ΙΖ\or ΙΗ\or ΙΘ\or Κ\or ΚΑ\or ΚΒ\or ΚΓ\or ΚΔ\or ΚΕ\or ΚΣ\or ΚΖ\or ΚΗ\or ΚΘ\else\thatexerror[AlphatheOutOfRange]\fi ʹ}

\def\constthe#1#2{#1}
\def\emptythe   {\constthe {}}
\def\bulletthe  {\constthe {$\knuthbullet$}}
\def\dashthe    {\constthe {--}}
\def\circthe    {\constthe {$\knuthcirc$}}
\def\starthe    {\constthe {$\knuthast$}}
\def\dotthe     {\constthe {$\knuthcdot$}}
\def\userthe    {\constthe {\userthesymbol}}

\def\userthesymbol{\thatexerror[userthesymbol never defined, yet I found userthe]}
\def\setuserthesymbol#1{\def\userthesymbol{#1}}

\def\oldthe#1{\oldstylize{\nthe#1}}

\def\thetypen   {{1}} % 1, 2, 3, ...
\def\thetypea   {{a}} % a, b, c, ...
\def\thetypei   {{i}} % i, ii, iii, ...
\def\thetypeg   {{g}} % α, β, γ, ...
\def\thetypeA   {{A}} % a, b, c, ...
\def\thetypeI   {{I}} % i, ii, iii, ...
\def\thetypeG   {{G}} % α, β, γ, ...
\def\thetypee   {{e}} % empty
\def\thetypeb   {{.}} % bullets
\def\thetypeo   {{o}} % circ
\def\thetyped   {{-}} % dashes
\def\thetypes   {{*}} % stars
\def\thetypeu   {{u}} % user-defined

% english the
\def\ENthe#1{\ifcase#1zero\or one\or two\or three\or four\or five\or six\or seven\or eight\or nine\or ten\or eleven\or twelve\or thirteen\or fourteen\or fifteen\or sixteen\or seventeen\or eighteen\or nineteen\or twenty\or \thatexerror[ENtheOutOfRange]\fi}

\def\humanthe#1{\ifnum#1>20\oldthe{#1}\else\ea\csname \ENthe#1word\endcsname\fi}
\def\humanmthe#1{\ifnum#1>20\oldthe{#1}\else\ea\csname \ENthe#1mword\endcsname\fi}
\def\humanfthe#1{\ifnum#1>20\oldthe{#1}\else\ea\csname \ENthe#1fword\endcsname\fi}

%%}}}

%%{{{ ThaTeX i18n 

\def\genderm{m}
\def\genderf{f}
\def\gendern{n}
\def\genderna{}

\def\sayone#1{\csname #1word\endcsname}
\def\saymany#1{\csname #1sword\endcsname}
\def\sayname#1{\csname #1name\endcsname}
\def\sayName#1{\csname #1Name\endcsname}
\def\sayNAME#1{\csname #1NAME\endcsname}
\def\saynames#1{\csname #1sname\endcsname}
\def\sayNames#1{\csname #1sName\endcsname}
\def\sayNAMES#1{\csname #1sNAME\endcsname}
\def\sayXname #1[#2]{\ifnum 1 = #1\relax\sayname{#2}\else\saynames{#2}\fi}
\def\sayXName #1[#2]{\ifnum 1 = #1\relax\sayName{#2}\else\sayNames{#2}\fi}
\def\sayXNAME #1[#2]{\ifnum 1 = #1\relax\sayNAME{#2}\else\sayNAMES{#2}\fi}

\def\humancount #1[#2]{%
\def\thisgender{#2}%
\ifx\thisgender\genderm\humanmthe{#1}%
\else\ifx\thisgender\genderf\humanfthe{#1}%
\else\humanthe{#1}\fi\fi
}
\def\humancountknowngender#1{% assumes \thisgender is set
\ifx\thisgender\genderm\humanmthe{#1}%
\else\ifx\thisgender\genderf\humanfthe{#1}%
\else\humanthe{#1}\fi\fi
}
\def\humancountword #1[#2]{%
\def\thisgender{\csname #2wordgender\endcsname}
\ifx\thisgender\genderm\humanmthe{#1}%
\else\ifx\thisgender\genderf\humanfthe{#1}%
\else\humanthe{#1}\fi\fi
\ifnum 1=#1\relax\sayone{#2}\else\saymany{#2}\fi
}
\def\humancountwordknowngender #1[#2]{% assumes \thisgender is set
\if\thisgender\genderm\humanmthe{#1}%
\else\if\thisgender\genderf\humanfthe{#1}%
\else\humanthe{#1}\fi\fi\spacechar
\ifnum 1=#1\relax\sayone{#2}\else\saymany{#2}\fi
}


% XXX: \DefNoun define capitalization variants of words
% because it seems easier this way than making just the first letter uppercase
\protected\def\DefNoun #1 #2: | #3 #4| #5 #6| {% #1: handle ; #2: gender ; #3: Singular ; #5: Plural ; #4,#6: hack to absorve whitespace
\def\TMP{#3}%
\ea\edef\csname #1Name\endcsname{\TMP}%
\ea\edef\csname #1Word\endcsname{\TMP}%
\lowercase{\def\TMP{#3}}%
\ea\edef\csname #1name\endcsname{\TMP}%
\ea\edef\csname #1word\endcsname{\TMP}%
\uppercase{\def\TMP{#3}}%
\ea\edef\csname #1NAME\endcsname{\TMP}%
\ea\edef\csname #1WORD\endcsname{\TMP}%
\def\TMP{#5}%
\ea\edef\csname #1sName\endcsname{\TMP}%
\ea\edef\csname #1sWord\endcsname{\TMP}%
\lowercase{\def\TMP{#5}}%
\ea\edef\csname #1sname\endcsname{\TMP}%
\ea\edef\csname #1sword\endcsname{\TMP}%
\uppercase{\def\TMP{#5}}%
\ea\edef\csname #1sNAME\endcsname{\TMP}%
\ea\edef\csname #1sWORD\endcsname{\TMP}%
\undef\TMP
\ea\edef\csname #1wordgender\endcsname{#2}% one of: m,f,n,(empty)
}
\def\DefTerm #1 #2{% #1: handle ; #2: singular
\ea\def\csname #1term\endcsname{#2}%
}
\def\DefNumWords #1 #2 #3 #4 {% #1: english ; #2: neutral ; #3: m ; #4: f
\ea\def\csname #1word\endcsname{#2}%
\ea\def\csname #1mword\endcsname{#3}%
\ea\def\csname #1fword\endcsname{#4}%
}
\def\DefNumWord #1 #2 {% #1: english ; #2: n/m/f
\DefNumWords #1 #2 #2 #2
}

%%}}}

%%{{{ ThaTeX datetime 
\def\today#1{\the\year#1\relax\ifnum\month<10 0\fi\the\month\relax#1\ifnum\day<10 0\fi\the\day\relax}
\def\humantime{
\innernewcount\hours
\innernewcount\minutes
\hours=\time
\divide\hours by 60
\minutes=\hours
\multiply\minutes by 60
\advance\minutes by -\time
\multiply\minutes by -1
\twodigitfmt{\the\hours}:\twodigitfmt{\the\minutes}
}
%%}}}

%%{{{ ThaTeX citations 

% high level interface, based on thatex-eplain
\def\printcitestart{\bgroup\ninessdc[}
\def\printcitefinish{]\egroup}
\def\printcitenote#1{\ninerm:~{\relax#1}}
\def\printbetweencitations#1{{\ninesl,}~}
\def\CITE#1{\eplaincite{#1}}
\def\CITEIN#1: #2\ENDCITEIN{\eplaincite[#2]{#1}}
\def\cite[#1]{\IfStrContains {#1} {: }
    {\CITEIN#1\ENDCITEIN}
    {\CITE{#1}}}

%%}}}

%%{{{ ThaTeX hyperlinks 

% beware not to define already existing colors (cryptic error message)
\definecolor{thatexcyan}{cmyk}{0.75,.0,.0,.5}
\definecolor{thatexpurple}{rgb}{0.444,.0,.666}
\definecolor{thatexorange}{rgb}{0.9,.3,.1}

\def\hlhrefopts{bstyle=U,bwidth=0,colormodel=,color=thatexpurple,zoom=2000}
\def\hlurlopts{bstyle=U,colormodel=,color=thatexorange,bcolor=.444 .0 .666,bwidth=.666}
\def\mark[#1]{\hldest{xyz}{\hlopts}{#1}}                % use: This is the chapter about set theory.\mark[settheory]
\def\href[#1|#2]{\hlstart{name}{\hlopts}{#1}#2\hlend}   % use: Studying \href[settheory|the theory of sets] we...
\def\urlrefas[#1|#2]{\hlstart{url}{\hlurlopts}{#1}{\tt #2}\hlend}     % use: Check \urlref[https://tsouanas.org/fmcbook] for more info.
\def\urlref[#1]{\urlrefas[#1|#1]}

%%}}}

%%{{{ ThaTeX refs 

% low-level, ignoring dir hierarchy system

% eplain based refs:
\protected\def\labcat#1#2#3{\definexref{#2}{#3}{#1}}% #1: category; #2: key; #3: value
\def\LABELRAW[#1=#2]{\labcat{\thislabelclass}{#1}{#2}}% #1: key; #2: value
\def\LABELPAGERAW[#1]{\xrdef{#1,page}}
\def\REFERRAW[#1]{\eplainrefn{#1}}
\def\REFERPAGERAW[#1:#2]{\ifnil#1\\\xref{#2,page}\else\xref[#1]{#2,page}\fi}
\def\REFERTEXTRAW[#1]{\csgroup\expanded{\xrlabel{#1}}}% no hyperlink, no refpwd

% TODO: from this point on macros should be based on the macros above as an interface

\def\REFER[#1]{\IfStrBeginsWith{#1}{/}
    {\eplainrefn{#1}}
    {\eplainrefn{\refpwd#1}}}
\def\LABEL[#1=#2]{\IfStrBeginsWith{#1}{/}
    {\labcat{\thislabelclass}{#1}{#2}}
    {\labcat{\thislabelclass}{\labpwd#1}{#2}}} % #1: key; #2: value
\def\ALTLABEL[#1=#2]{\IfStrBeginsWith{#1}{/}
    {\eplaineqdefn[#2]{#1}}
    {\eplaineqdefn[#2]{\labpwd#1}}}
\def\LABELPAGE[#1]{\IfStrBeginsWith{#1}{/}
    {\xrdef{#1,page}}
    {\xrdef{\labpwd#1,page}}}
\def\ALTREFER[#1]{\IfStrBeginsWith{#1}{/}
    {\eplaineqref{#1}}
    {\eplaineqref{\refpwd#1}}}
\def\ALTREFERNUDE[#1]{\IfStrBeginsWith{#1}{/}
    {\eplaineqrefn{#1}}
    {\eplaineqrefn{\refpwd#1}}}
\def\REFERPAGE[#1:#2]{\ifnil#1\\\xref\else\xref[#1]\fi
    \IfStrBeginsWith{#2}{/}
        {{#2,page}}
        {{\refpwd#2,page}}}
\def\REFERTEXT[#1]{\IfStrBeginsWith{#1}{/}
    {\expanded{\csname\xrlabel{#1}\endcsname}}
    {\expanded{\csname\xrlabel{\refpwd#1}\endcsname}}}% no hyperlink


\def\reftext[#1]{\REFERTEXT[#1]}

\def\labp[#1=#2]{\LABEL[#1=#2]\labpage[#1]}% #1: key; #2: value
\def\labpage[#1]{\LABELPAGE[#1]}
\def\refpage[#1]{\REFERPAGE[:#1]}
\def\refpabbr[#1]{\REFERPAGE[\pageabbr:#1]}
\def\refpword[#1]{\REFERPAGE[\pageword:#1]}

\def\mathref[#1]{\ALTREFER[#1]}
\def\mathrefnude[#1]{\ALTREFERNUDE[#1]}
\def\axiomref[#1]{\ALTREFER[ax:#1]}
\def\axiomrefnude[#1]{\ALTREFERNUDE[ax:#1]}
\let\mref=\mathref
\let\mrefn=\mathrefnude
\let\axref=\axiomref
\let\axrefn=\axiomrefnude

\def\labchapof[#1=#2]{\LABEL[#1,chap=#2]}
\def\labsecof[#1=#2]{\LABEL[#1,sec=#2]}
\def\labnoteof[#1=#2]{\LABEL[#1,note=#2]}
\def\refchapof[#1]{\REFER[#1,chap]}
\def\refsecof[#1]{\REFER[#1,sec]}
\def\refnoteof[#1]{\REFER[#1,note]}

\def\refmany[#1]{\csname\getproperty{\xrlabel{#1}}{class}sname\endcsname}% eg: theorems (to be used with further \reftag's)
\def\Refmany[#1]{\csname\getproperty{\xrlabel{#1}}{class}sName\endcsname}% like above but capitalized

% ThaTeX lab/refs: from this point on only the interface above should be used
% XXX: this is not yet true for auto/math/decotags

\def\labsuffix #1 [#2=#3]{\LABEL[#2,#1=#3]}
\def\labidof        {\labsuffix id }
\def\labsmart       {\labsuffix smart }
\def\labnear        {\labsuffix near }
\def\labtag         {\labsuffix tag }
\def\labfull        {\labsuffix full }
\def\labtitle       {\labsuffix title }
\def\labformal      {\labsuffix formal }

\def\labraw[#1=#2]{\labcat{raw}{#1}{#2}}
\def\labtagofid[#1=#2]{\labraw[#1,tag=#2]} % usually sets the tag of an id
\def\lablabelofid[#1=#2]{\labraw[#1,label=#2]} % usually sets the label of an id
\def\labpageofid[#1]{\LABELPAGERAW[#1]}
\def\refidof[#1]{\REFER[#1,id]} % this should normally return the tag
\def\reftagofid[#1]{\REFERRAW[#1,tag]}
\def\refpageofid[#1]{\REFERPAGERAW[:#1]}
\def\refpabbrofid[#1]{\REFERPAGERAW[\pageabbr:#1]}
\def\refpwordofid[#1]{\REFERPAGERAW[\pageword:#1]}
\def\reflabelofid[#1]{\REFERTEXTRAW[#1,label]} % this should normally return the label

\def\refsuffix #1 [#2]{\REFER[#2,#1]}
\def\refsmart       {\refsuffix smart }
\def\refnear        {\refsuffix near }
\def\reftag         {\refsuffix tag }
\def\reffull        {\refsuffix full }
\def\reftitle       {\refsuffix title }
\def\refformal      {\refsuffix formal }
\def\refcountnum    {\refsuffix count }
\def\refcountthe    {\refsuffix count,show }
\def\refcountn      {\refsuffix count,human }
\def\refcount       {\refsuffix count,human,full }
\def\refhumanm      {\refsuffix count,humanm }
\def\refhumanf      {\refsuffix count,humanf }
\def\refnoun        {\refsuffix count,noun }
\def\refnounone     {\refsuffix count,nouns }
\def\refnounmany    {\refsuffix count,nounp }
\def\refcountfar    {\refsuffix count,human,full,far }

\def\refcountnfull[#1]{\refcountn[#1] \refnoun[#1]}% shows the pretty numeral titled

% defaults:
\let\ref=\refsmart
\let\lab=\LABEL


% configurators
\def\setlabelclass#1{\def\thislabelclass{#1}}
\def\eachitem #1 {\setlabelclass{#1}}%
\def\labelclass #1 {\setlabelclass{#1}}%

\setlabelclass{}

\def\rootdir{/}
\let\labpwd=\rootdir
\let\refpwd=\rootdir
\def\labpath#1\\{/\if/#1\else\ea\eatfst\labpwd#1/\fi}% explanation below
\def\refpath#1\\{/\if/#1\else\ea\eatfst\refpwd#1/\fi}
% XXX: delete this DATED \labpath / \refpath explained ?:
% \labpath#1\\ prints path #1 (which can be absolute or relative) as an absolute path
% 1: /%                            print / because all absolute paths start with a slash
% 2: \if/#1%                       if #1 starts with / (it is an absolute path) print it without its initial slash because we already have one from line 1 of this def and we are done
% 3: \else%                        #1 is a relative path
% 4: \ea\eatfst\labpwd%   print \labpwd without its initial slash (because we already have one from line 1 of this def)
%                                  now we must join the relative path (#1) part to what we have already printed
%                                  but we must be careful we must use a slash to join the two parts iff labpws is not root (/):
%                                  if labpwd is root (/) then do not use a slash to join
%                                  labpwd is (/something) which we have already printed (line 1 and 4 of this def) so print a slash to join the two parts:
% 5: \ifx\labpwd\rootdir\else/\fi% done with the glue
% 8: #1\fi}%                        finally print the second part and we're done
% CD commands for internal use
\protected\def\labCD#1{\edef\labpwd{\labpath#1\\}}
\protected\def\refCD#1{\edef\refpwd{\refpath#1\\}}
\def\labrefCD#1{\labCD{#1}\refCD{#1}}
% cd commands for user
\def\labcd #1 {\ifnil#1\\\thatexerror[CannotChangeDirToEmptyThisIsNotReallyUnixYouAreHomelessHere]\fi\labCD{#1}}
\def\refcd #1 {\ifnil#1\\\thatexerror[CannotChangeDirToEmptyThisIsNotReallyUnixYouAreHomelessHere]\fi\refCD{#1}}
\def\labrefcd #1 {\labrefCD{#1}}
\def\labcdroot{\labCD{/}}
\def\refcdroot{\refCD{/}}
\def\labrefcdroot{\labrefCD{/}}
\let\cdroot=\labrefcdroot
%%}}}

%%{{{ ThaTeX tlist 

\newcount\tlistlevel  \tlistlevel=0
\newdimen\tlistindent \tlistindent=1.333\parindent
\newif\iftliststarted \tliststartedfalse % has this tlist produced any tlitems?
\newif\iftlistcompact \tlistcompactfalse
\newif\iftlistjustskipped \tlistjustskippedfalse

\def\compactlists{\global\tlistcompacttrue}
\def\spaciouslists{\global\tlistcompactfalse}

\Newskip tlist=\smallskipamount; % space above and below \tlist...\endtlist
\Newskip tlitem=\tinyskipamount;  % space above \tlitem's (except first)

\def\tlitemspace{\enspace\thinspace}

\def\neartag#1{#1}% usually overwritten by thatexbook
\def\fartag#1{#1}% usually overwritten by thatexbook

\def\settlistshowtaga#1{\def\tlistshowtaga{#1}}
\def\settlistshowtagz#1{\def\tlistshowtagz{#1}}
\def\tlistshowtag#1{\tlistshowtaga{#1}\tlistshowtagz}
\settlistshowtaga {}
\settlistshowtaga {}

\protected\def\tlist #1:#2 {%
% 1: tag style
% 2: name (ignored for now)
\ifnil#2\\\else\thatexwarn[tlist was given a name (#2) but this does nothing for now.]\fi
    \par\tlistskip
    \global\tlistjustskippedtrue
    \bgroup
    \ifnil#1\\\def\tlitemtag{$\bullet$}\else\def\tlitemtag{#1}\fi
    \def\iftoptlist{\ifnum\tlistlevel=0\relax}
    %% internal use
    \def\tlistskipifnotcompact{\iftlistcompact\else\tlitemskip\fi}
    % customly tagged tlitem:
    \def\ctlitem##1{\par
        \iftliststarted\tlistskipifnotcompact\fi
        \tliststartedtrue
        \noindent\llap{##1\tlitemspace}\ignorespaces
        \global\tlistjustskippedfalse}
    \def\ctlitemlab##1##2{\ctlitem{##1}\ifvanishes##2\\\else\labp[##2=##1]\labsmart[##2=\fartag{##1}]\labnear[##2=\neartag{##1}]\fi}
    % default tlitem
    \def\tlitem{\ctlitem{\tlitemtag}}
    % tlitem that works with empty or nonempty tag
    \def\tlitemauto##1##2{\ifnil##1\\\tlitem\else\ctlitemlab{##1}{##2}\fi}
    %% user customization
    \def\spaciouslist{\tlistcompactfalse}
    \def\compactlist{\tlistcompacttrue}
    \def\dresstags with ##1_##2 {\settlistshowtaga{##1}\settlistshowtagz{##2}}%
    \def\nakedtags{\dresstags with {}_{} }
    \def\withtag ##1 {\def\tlitemtag{##1}}
    \def\notags{\withtag {} }
    % normal use
    \def\li ##1:##2 {\tlitemauto{##1}{##2}}
    %% setup: nest, count, etc.
    \advance\leftskip by\tlistindent
    \advance\tlistlevel by 1\relax
}

\def\endtlist{%
% if it has a name, create count and page labels
\endgraf
\egroup
% avoid double tlistskip
\iftlistjustskipped\else\tlistskip\global\tlistjustskippedtrue\fi
\par
\noindent
}

%%}}}

%%{{{ ThaTeX enum 

\newcount \enumcount        % not really used for counting
\newcount \enumicount
\newcount \enumiicount
\newcount \enumiiicount
\newcount \enumiiiicount
\newcount \enumiiiiicount

\newcount \enumlevel    \enumlevel = 0
\newcount \maxenumlevel \maxenumlevel = 5
\newcount \minenumlevel \minenumlevel = 0

\def\enumsep        {}% this should always be empty
\def\enumisep       {}
\def\enumiisep      {.}
\def\enumiiisep     {.}
\def\enumiiiisep    {.}
\def\enumiiiiisep   {.}

\def\enumtype      {{e}}
\def\enumitype     {{1}}
\def\enumiitype    {{a}}
\def\enumiiitype   {{i}}
\def\enumiiiitype  {{g}}
\def\enumiiiiitype {{1}}

\def\setenumshowtaga#1{\def\enumshowtaga{#1}}
\def\setenumshowtagz#1{\def\enumshowtagz{#1}}
\def\enumshowtag#1{\enumshowtaga{#1}\enumshowtagz}
\setenumshowtaga {(}
\setenumshowtagz {)}

\def\enumncount[#1]{\ifcase#1\enumcount\or\enumicount\or\enumiicount\or\enumiiicount\or\enumiiiicount\or\enumiiiiicount
                    \else\thatexerror[enumlevel off limits]\fi}
\def\enumnsep  [#1]{\ifcase#1\enumsep  \or\enumisep  \or\enumiisep  \or\enumiiisep  \or\enumiiiisep  \or\enumiiiiisep
                    \else\thatexerror[enumlevel off limits]\fi}
\def\enumntype [#1]{\ifcase#1\enumtype \or\enumitype \or\enumiitype \or\enumiiitype \or\enumiiiitype \or\enumiiiiitype
                    \else\thatexerror[enumlevel off limits]\fi}

\def\setenumntype#1#2{%
\ifcase#1\def\enumtype      {{#2}}\or
         \def\enumitype     {{#2}}\or
         \def\enumiitype    {{#2}}\or
         \def\enumiiitype   {{#2}}\or
         \def\enumiiiitype  {{#2}}\or
         \def\enumiiiiitype {{#2}}\else
         \thatexerror[enumlevel off limits]\fi}
\def\setthisenumtype#1{\setenumntype{\enumlevel}{#1}}

\def\setenumnsep#1#2{%
\ifcase#1\def\enumsep      {{#2}}\or
         \def\enumisep     {{#2}}\or
         \def\enumiisep    {{#2}}\or
         \def\enumiiisep   {{#2}}\or
         \def\enumiiiisep  {{#2}}\or
         \def\enumiiiiisep {{#2}}\else
         \thatexerror[enumlevel off limits]\fi}
\def\setthisenumsep#1{\setenumnsep{\enumlevel}{#1}}

%%{{{ \enumnthe-- 
\def\enumnthe#1#2{%
    \ifcase#1\ifx\enumtype      \thetypen \nthe{#2}\else
             \ifx\enumtype      \thetypea \athe{#2}\else
             \ifx\enumtype      \thetypei \ithe{#2}\else
             \ifx\enumtype      \thetypeg \gthe{#2}\else
             \ifx\enumtype      \thetypeA \Athe{#2}\else
             \ifx\enumtype      \thetypeI \Ithe{#2}\else
             \ifx\enumtype      \thetypeG \Gthe{#2}\else
             \ifx\enumtype      \thetypee \emptythe{#2}\else
             \ifx\enumtype      \thetypeb \bulletthe{#2}\else
             \ifx\enumtype      \thetypeo \circthe{#2}\else
             \ifx\enumtype      \thetyped \dashthe{#2}\else
             \ifx\enumtype      \thetypes \starthe{#2}\else
             \ifx\enumtype      \thetypeu \userthe{#2}\else
             \thatexerror[Unknown enumtype.]
             \fi\fi\fi\fi\fi\fi\fi\fi\fi\fi\fi\fi\fi
        \or  \ifx\enumitype     \thetypen \nthe{#2}\else
             \ifx\enumitype     \thetypea \athe{#2}\else
             \ifx\enumitype     \thetypei \ithe{#2}\else
             \ifx\enumitype     \thetypeg \gthe{#2}\else
             \ifx\enumitype     \thetypeA \Athe{#2}\else
             \ifx\enumitype     \thetypeI \Ithe{#2}\else
             \ifx\enumitype     \thetypeG \Gthe{#2}\else
             \ifx\enumitype     \thetypee \emptythe{#2}\else
             \ifx\enumitype     \thetypeb \bulletthe{#2}\else
             \ifx\enumitype     \thetypeo \circthe{#2}\else
             \ifx\enumitype     \thetyped \dashthe{#2}\else
             \ifx\enumitype     \thetypes \starthe{#2}\else
             \ifx\enumitype     \thetypeu \userthe{#2}\else
             \thatexerror[Unknown enumtype.]
             \fi\fi\fi\fi\fi\fi\fi\fi\fi\fi\fi\fi\fi
        \or  \ifx\enumiitype    \thetypen \nthe{#2}\else
             \ifx\enumiitype    \thetypea \athe{#2}\else
             \ifx\enumiitype    \thetypei \ithe{#2}\else
             \ifx\enumiitype    \thetypeg \gthe{#2}\else
             \ifx\enumiitype    \thetypeA \Athe{#2}\else
             \ifx\enumiitype    \thetypeI \Ithe{#2}\else
             \ifx\enumiitype    \thetypeG \Gthe{#2}\else
             \ifx\enumiitype    \thetypee \emptythe{#2}\else
             \ifx\enumiitype    \thetypeb \bulletthe{#2}\else
             \ifx\enumiitype    \thetypeo \circthe{#2}\else
             \ifx\enumiitype    \thetyped \dashthe{#2}\else
             \ifx\enumiitype    \thetypes \starthe{#2}\else
             \ifx\enumiitype    \thetypeu \userthe{#2}\else
             \thatexerror[Unknown enumtype.]
             \fi\fi\fi\fi\fi\fi\fi\fi\fi\fi\fi\fi\fi
        \or  \ifx\enumiiitype   \thetypen \nthe{#2}\else
             \ifx\enumiiitype   \thetypea \athe{#2}\else
             \ifx\enumiiitype   \thetypei \ithe{#2}\else
             \ifx\enumiiitype   \thetypeg \gthe{#2}\else
             \ifx\enumiiitype   \thetypeA \Athe{#2}\else
             \ifx\enumiiitype   \thetypeI \Ithe{#2}\else
             \ifx\enumiiitype   \thetypeG \Gthe{#2}\else
             \ifx\enumiiitype   \thetypee \emptythe{#2}\else
             \ifx\enumiiitype   \thetypeb \bulletthe{#2}\else
             \ifx\enumiiitype   \thetypeo \circthe{#2}\else
             \ifx\enumiiitype   \thetyped \dashthe{#2}\else
             \ifx\enumiiitype   \thetypes \starthe{#2}\else
             \ifx\enumiiitype   \thetypeu \userthe{#2}\else
             \thatexerror[Unknown enumtype.]
             \fi\fi\fi\fi\fi\fi\fi\fi\fi\fi\fi\fi\fi
        \or  \ifx\enumiiiitype  \thetypen \nthe{#2}\else
             \ifx\enumiiiitype  \thetypea \athe{#2}\else
             \ifx\enumiiiitype  \thetypei \ithe{#2}\else
             \ifx\enumiiiitype  \thetypeg \gthe{#2}\else
             \ifx\enumiiiitype  \thetypeA \Athe{#2}\else
             \ifx\enumiiiitype  \thetypeI \Ithe{#2}\else
             \ifx\enumiiiitype  \thetypeG \Gthe{#2}\else
             \ifx\enumiiiitype  \thetypee \emptythe{#2}\else
             \ifx\enumiiiitype  \thetypeb \bulletthe{#2}\else
             \ifx\enumiiiitype  \thetypeo \circthe{#2}\else
             \ifx\enumiiiitype  \thetyped \dashthe{#2}\else
             \ifx\enumiiiitype  \thetypes \starthe{#2}\else
             \ifx\enumiiiitype  \thetypeu \userthe{#2}\else
             \thatexerror[Unknown enumtype.]
             \fi\fi\fi\fi\fi\fi\fi\fi\fi\fi\fi\fi\fi
        \or  \ifx\enumiiiiitype \thetypen \nthe{#2}\else
             \ifx\enumiiiiitype \thetypea \athe{#2}\else
             \ifx\enumiiiiitype \thetypei \ithe{#2}\else
             \ifx\enumiiiiitype \thetypeg \gthe{#2}\else
             \ifx\enumiiiiitype \thetypeA \Athe{#2}\else
             \ifx\enumiiiiitype \thetypeI \Ithe{#2}\else
             \ifx\enumiiiiitype \thetypeG \Gthe{#2}\else
             \ifx\enumiiiiitype \thetypee \emptythe{#2}\else
             \ifx\enumiiiiitype \thetypeb \bulletthe{#2}\else
             \ifx\enumiiiiitype \thetypeo \circthe{#2}\else
             \ifx\enumiiiiitype \thetyped \dashthe{#2}\else
             \ifx\enumiiiiitype \thetypes \starthe{#2}\else
             \ifx\enumiiiiitype \thetypeu \userthe{#2}\else
             \thatexerror[Unknown enumtype.]
             \fi\fi\fi\fi\fi\fi\fi\fi\fi\fi\fi\fi\fi
      \else\thatexerror[enumlevel too high]\fi
}
%%}}}

\def\enumlabel{}
\def\setenumlabel#1{\def\enumlabel{#1}}

\def\thisenumprefix{}
\def\thisenumsep{\enumnsep[\enumlevel]}
\def\thisenitemtag{\enumnthe{\enumlevel}{\thisenumcount}}
\def\thisenitemfulltag{\thisenumprefix\thisenumsep\thisenitemtag}
\def\thisenitemfartag{\fartag{\thisenitemfulltag}}
\def\showenitemtag{\enumshowtag{\thisenitemtag}}
\def\showenitemfulltag{\enumshowtag{\thisenitemfulltag}}
\def\showenitemfartag{\enumshowtag{\thisenitemfartag}}
\def\thisenumcount{\enumncount[\enumlevel]}
\def\countinc{\global\advance\thisenumcount by 1\relax}
\def\countreset{\global\thisenumcount=0\relax}

\def\enumenter{%
\ifnum\enumlevel=\maxenumlevel\thatexerror[cannot enumenter past level \the\maxenumlevel]\fi
\bgroup
\setenumlabel{}%
\setlabelclass{item}%
\ifnil\thisenitemtag\\\else\edef\thisenumprefix{\thisenitemfulltag}\fi
\global\advance\enumlevel by 1\relax
\countreset
}

\def\enumleave{%
\ifnum\enumlevel=\minenumlevel\thatexerror[cannot enumleave past level \the\minenumlevel]\fi
\def\dolabword##1{\ifvanishes\thislabelclass\\\else\labword##1[\enumlabel=\thislabelclass]\fi}%
\ifvanishes\enumlabel\\\else\labcount[\enumlabel=\thisenumcount]\labnoteof[\enumlabel=\structtag]\dolabword{\thisenumcount}\fi
\countreset
\global\advance\enumlevel by -1\relax
\egroup
}

% high-level interface

\def\enum #1:#2 {\enumenter
% 1: enum type
% 2: enum name/label
\ifnil#1\\\else\setthisenumtype{#1}\fi
\ifnil#2\\\else\setenumlabel{#2}\fi
\def\dresstags with ##1_##2 {\setenumshowtaga{##1}\setenumshowtagz{##2}}%
\def\nakedtags{\dresstags with {}_{} }
\def\sepwith ##1 {\setthisenumsep{##1}}%
\def\whereu is ##1 {\setuserthesymbol{##1}\def\userthe{\constthe{\userthesymbol}}}%
\def\nn{\countinc}%
\def\ti{\thisenitemtag}%
\def\TI{\thisenitemfulltag}%
\def\si{\showenitemtag}%
\def\SI{\showenitemfulltag}%
\def\ei{\nn\si}%
\def\EI{\nn\SI}%
}
\def\endenum{\enumleave
}

%% more (i)lab/(i)ref macros
\def\countthe#1{\oldthe{#1}}

\def\labcount[#1=#2]{%
\bgroup
% XXX: this could pollute the namespace, trying it in a group now
\def\thisgender{\csname \thislabelclass wordgender\endcsname}
% XXX: these might work better with specialized \lab... like the \ref... ones.
\LABEL[#1,count=\nthe{#2}]%
\LABEL[#1,count,show=\countthe{#2}]%
\LABEL[#1,count,human=\humancountknowngender{#2}]%  human count (known gender)
\LABEL[#1,count,humanm=\humanmthe{#2}]% human count (for m. nouns)
\LABEL[#1,count,humanf=\humanfthe{#2}]% human count (for f. nouns)
\edef\thiscountfull{\humancountwordknowngender {#2}[\thislabelclass]}
\ifcsname\thislabelclass word\endcsname
\LABEL[#1,gender=\thisgender]%
\LABEL[#1,count,human,full=\thiscountfull]%
% XXX: these two assume paragtag which is only defined in thatexbook!
\LABEL[#1,count,human,full,far=\thiscountfull\ (\paragtag)]%
\LABEL[#1,smart=\thiscountfull\ (\paragtag)]%
\fi
\egroup
}

\def\labword#1[#2=#3]{\ifvanishes#2\\\thatexerror[empty name on setting labword]\fi
% 1: count
% 2: name of count
% 3: words (noun) to describe counted items
\LABEL[#2,count,nouns=\csname #3word\endcsname]%
\LABEL[#2,count,nounp=\csname #3sword\endcsname]%
\ifnum #1=1\relax
\LABEL[#2,count,noun=\csname #3word\endcsname]%
\else
\LABEL[#2,count,noun=\csname #3sword\endcsname]%
\fi
}

%%}}}

%%{{{ ThaTeX elist: tlist + enum 

\def\elist #1:#2 {%
% 1: numeral style of enum
% 2: name of enum and tlist
%% always open tlist nameless and without itag
\tlist :#2
\enum #1:#2
\def\li ##1:##2 {\nn\tlitemauto{\si}{##2}}%
\def\LI ##1:##2 {\nn\tlitemauto{\SI}{##2}}%
}

\def\endelist{\endenum\endtlist}
%%}}}

%%{{{ ThaTeX math counting and refs/labs

\newcount\savedmcount
\newif\ifloadmcount
\global\loadmcountfalse

\def\reseteqnumber{\global\eqnumber=0\relax}
\def\resetsavedmcount{\global\savedmcount=0\relax}
\def\resetmcount{\reseteqnumber\resetsavedmcount}
\def\loadeqnumber{\global\eqnumber=\savedmcount\relax}
\def\saveeqnumber{\ifnum\eqnumber>0\global\savedmcount=\eqnumber\relax\fi}
\def\fixmcount{\aftergroup\loadeqnumber}

\def\manuinvtag[#1=#2]{{\let\eqconstruct=\ideqconstruct
                        \ALTLABEL[#1=\neartag{#2}]\empty
                        \let\eqconstruct=\fareqconstruct
                        \ALTLABEL[#1,full=\fartag{#2}]}}
\def\manutag[#1=#2]{\tag{\manuinvtag[#1=#2]#2}}
\def\tmanutag[#1=#2]{\textrm{\labp[#1=#2]\labsmart[#1=\fartag{#2}]{(#2)}}}
\def\axiominvtag[#1=#2]{{\let\eqconstruct=\ideqconstruct
                         \ALTLABEL[ax:#1=\neartag{#2}]}}
\def\axiomtag[#1=#2]{\tag{\axiominvtag[#1=#2]#2}}
% XXX: \def\axiomtag[#1=#2]{\textrm{\labp[ax:#1=#2]{(#2)}}}

\def\autoinvtag[#1]{{\let\eqconstruct=\ideqconstruct
                     \eplaineqdefn{\labpwd#1}\empty
                     \labsmart[#1=\fartag{\the\eqnumber}]}}
\def\autotag[#1]{\tag{\autoinvtag[#1]\the\eqnumber}}
\def\tautotag[#1]{\global\advance\eqnumber by 1\relax
                  \tmanutag[#1=\the\eqnumber]}

\def\decotag[#1=#2 with #3_#4]{\manutag[#1={#3\csgroup{\xrlabel{\refpwd#2}}#4}]}

% abbreviations:
\let\mtag  = \manutag
\let\atag  = \autotag
\let\axtag  = \axiomtag
\let\mitag = \manuinvtag
\let\aitag = \autoinvtag
\let\tatag = \tautotag
\let\tmtag = \tmanutag
\let\dtag  = \decotag

%%}}}

%%{{{ ThaTeX draft 
\definecolor{todocolor}{rgb}{.0,.5,.75}
\definecolor{todocolorinv}{rgb}{1,1,1}
\def\TODO#1.{{}\eop\noi{\llap{\colorboxfb{todocolorinv}{todocolor}{\sfmbf~~TODO~~}\enspace}{\color{todocolor}\sf #1}}\eop}
\def\hiddenTODO{{}}
%%}}}

%%{{{ ThaTeX text 
\let\eop=\endgraf
\let\noi=\noindent
\def\emph#1{{\it #1}}
\def\strikeout#1{\ensuremath{\tikz[baseline] \node [strike out,draw,anchor=text,inner sep=0pt,text=black]{$\text{#1}$};}}
\def\yearof#1{\oldstylize{#1}}
\definecolor{alert}{rgb}{.9,.0,.0}
\definecolor{alertR}{rgb}{.9,.0,.0}
\definecolor{alertG}{rgb}{.0,.7,.0}
\definecolor{alertB}{rgb}{.0,.0,.999}
\definecolor{normal}{rgb}{.0,.0,.0}
\definecolor{spoiler}{rgb}{.666,.666,.666}
\definecolor{faded}{rgb}{.666,.666,.666}
\definecolor{veryfaded}{rgb}{.75,.75,.75}
\definecolor{veryveryfaded}{rgb}{.888,.888,.888}
\definecolor{trueK}{rgb}{0,0,0}
\definecolor{trueR}{rgb}{1,0,0}
\definecolor{trueG}{rgb}{0,1,0}
\definecolor{trueB}{rgb}{0,0,1}
\DefStyle aR {\color{alertR}}
\DefStyle aG {\color{alertG}}
\DefStyle aB {\color{alertB}}
\let\aR=\aRstyle
\let\aG=\aGstyle
\let\aB=\aBstyle
\def\alert#1{{\color{alert}#1}}
\def\alertR#1{{\color{alertR}#1}}
\def\alertG#1{{\color{alertG}#1}}
\def\alertB#1{{\color{alertB}#1}}
\def\trueR#1{{\color{trueR}#1}}
\def\trueG#1{{\color{trueG}#1}}
\def\trueB#1{{\color{trueB}#1}}
\let\alerta=\alertR
\let\alertb=\alertB
\let\alertc=\alertG
\def\faded#1{{\color{faded}#1}}
\def\veryfaded#1{{\color{veryfaded}#1}}
\def\veryveryfaded#1{{\color{veryveryfaded}#1}}
\defcompose\ufaded\faded\underline
\defcompose\uveryfaded\veryfaded\underline
\defcompose\uveryveryfaded\veryveryfaded\underline
\def\standout#1\endstandout{%
\endgraf
\bigskip
\centerline{#1}
\bigskip
\endgraf
\noindent
}
\def\quote{%
\endgraf
\medskip
\goodbreak
\advance \leftskip 2\parindent \advance \rightskip 2\parindent%
\begingroup
\noindent
}
\def\quotepar{\quote\indent}
\def\endquote{%
\endgraf
\endgroup
\medskip
\goodbreak
\advance \leftskip -2\parindent \advance \rightskip -2\parindent%
\noindent
}
% text alignments: centering, righting, lefting
\def\centeringmargin#1 #2\endcenteringmargin{%
\par
\begingroup
\leftskip=#1 plus 1fil
\rightskip=\leftskip
\parindent=0pt
\parfillskip=0pt
#2%
\par
\endgroup
}
\def\centering#1\endcentering{\centeringmargin 0pt #1\endcenteringmargin}
\def\rightingmargin#1 #2\endrightingmargin{%
\par
\begingroup
\leftskip=#1 plus 1fil
\rightskip=0pt
\parindent=0pt
\parfillskip=0pt
#2%
\par
\endgroup
}
\def\righting#1\endrighting{\rightingmargin 0pt #1\endrightingmargin}
\def\leftingmargin#1 #2\endleftingmargin{%
\par
\begingroup
\rightskip=#1 plus 1fil
\leftskip=0pt
\parindent=0pt
\parfillskip=0pt
#2%
\par
\endgroup
}
\def\lefting#1\endlefting{\leftingmargin 0pt #1\endleftingmargin}

\def\leftindent{%
\endgraf
\smallskip
\advance \leftskip 2\parindent%
\begingroup
\noindent
}
\def\endleftindent{%
\endgraf
\endgroup
\smallskip
\advance \leftskip -2\parindent%
\noindent
}
\def\spoken#1\endspoken{\quote\emph{#1}\endquote}
\def\dialogue#1\enddialogue{{\tlist: \def\say{\li --: }\def\who ##1: {\li {##1:}: }#1\endtlist}}
\def\tablerule{\noalign{\smallskip\hrule\smallskip}}
\def\tablethickrule{\noalign{\smallskip\hrule\hrule\hrule\smallskip}}
\def\tabletoprule{\noalign{\hrule\smallskip}}
\def\tablebottomrule{\noalign{\smallskip\hrule}}
\def\tool#1{{\sf #1}}
\def\symquote#1{`\,#1\,'}
\def\symqquote#1{``\,#1\,''}
\def\wordquote#1{<<#1>>}
\def\singlequote#1{`#1'}
\def\doublequote#1{``#1''}
\def\mathsymquote#1{\text{\symquote{$#1$}}}
\def\mathsymqquote#1{\text{\symqquote{$#1$}}}
\def\textwq#1{\text{\wq{#1}}}
\let\symq=\symquote
\let\symqq=\symqquote
\let\wq=\wordquote
\let\sq=\singlequote
\let\dq=\doublequote
\let\msymq=\mathsymquote
\let\msymqq=\mathsymqquote
\def\utter#1{\wq{#1}}
\def\flwq#1{\llap{<<}#1\rlap{>>}}% wq with floating quotes for use in \quote's etc.
\def\dotspc{{.} }% XXX: to work with \remark, etc.
\def\versus{vs\dotspc}% XXX: to work with \remark, etc.
\let\vs=\versus
\def\paraccenta#1{% XXX: works well with a's
$
\mathord{%
{}\kern-.3ex%
{{}^{{}^{{}_(}}}%
\kern-.6ex%
\text{#1}%
\kern-.4ex%
{{}^{{}^{{}_)}}}%
\kern-.48ex{}%
}
$%
}
%%}}}

%%{{{ ThaTeX aux files 
\def\safereadfile#1{
\testfileexistence{#1}%
\iffileexists%
\input \jobname.#1%
\fi%
}
%%}}}

%%{{{ ThaTeX ifs 

% versions of book:
\newif\ifprint\printfalse  % printed
\newif\ifdraft\drafttrue

%%}}}

%%{{{ ThaTeX titles and page decorations 

\newif\iftitlepage          \titlepagefalse
\newtoks\titlepagehead      \titlepagehead={\hfil}
\newtoks\titlepagefoot      \titlepagefoot={\hfil\twelverm---~{\oldstylize{\number\pageno}}~---\hfil}
\newtoks\toptitle           \toptitle={\hfil}
\newtoks\topmarker          \topmarker={\hfil}
\newtoks\oddpagehead        \oddpagehead={\hfil}
\newtoks\evenpagehead       \evenpagehead={\hfil}
\newtoks\evenpagefoot       \evenpagefoot={\hfil}
\newtoks\oddpagefoot        \oddpagefoot={\hfil}

% defaults
\oddpagehead={\nineit\the\toptitle\hfil\llap{\enspace\ninebf\number\pageno}}
\evenpagehead={\rlap{\ninebf\number\pageno\enspace}\hfil\nineit\the\topmarker}
\evenpagefoot={\hfil}
\oddpagefoot={\hfil}

\headline={%
\iftitlepage\the\titlepagehead
\else\ifodd\pageno\the\oddpagehead
\else\the\evenpagehead\fi\fi
}

\footline={%
\iftitlepage\the\titlepagefoot
\global\titlepagefalse
\else\ifodd\pageno\the\oddpagefoot
\else\the\evenpagefoot\fi\fi
}

\def\nohead{\headline={}}
\def\nofoot{\footline={}}

\def\nodecorspage{\nohead\nofoot}
\def\blankpage{{\nodecorspage\null\eject}}
\def\oddpage{\ifodd\pageno\else\blankpage\fi}
\def\makenextpageodd{\ifodd\pageno\eject\null\blankpage\fi}
\def\partpage{\oddpage\nodecorspage}

%%}}}

%%{{{ ThaTeX inserts 

%%{{{ amscaptions: code taken form amsppt.sty
\makeatletter
\def\first@#1#2\end{#1}
\def\true@{TT}
\def\false@{TF}
\def\empty@{}
\begingroup  \catcode`\-=3
\long\gdef\notempty#1{%
  \ea\ifx\first@#1-\end-\empty@ \false@\else \true@\fi}
\endgroup
\def\captionfont{}
\def\topcaption#1#2\endcaption{%
  {\dimen@\hsize \advance\dimen@-\captionwidth@
   \rm\raggedcenter@ \advance\leftskip.5\dimen@ \rightskip\leftskip
  {\captionfont#1}%
  \if\notempty{#2}\if\notempty{#1}.\enspace\fi\ignorespaces#2\fi
  \endgraf}\nobreak\bigskip}
\def\botcaption#1#2\endcaption{%
  \nobreak\bigskip
  \setboxz@h{\captionfont#1\if\notempty{#2}\if\notempty{#1}.\enspace\fi
    \rm\ignorespaces#2\fi}%
  {\dimen@\hsize \advance\dimen@-\captionwidth@
   \leftskip.5\dimen@ \rightskip\leftskip
   \noindent \ifdim\wdz@>\captionwidth@
   \else\hfil\fi
  {\captionfont#1}%
  \if\notempty{#2}\if\notempty{#1}.\enspace\fi\rm\ignorespaces#2\fi\endgraf}}
\makeatother
%%}}}

\let\amstopinsert=\topinsert
\let\amsmidinsert=\midinsert
\let\amsendinsert=\endinsert
\def\topinsert{\amstopinsert\noindent}
\def\midinsert{\goodbreak\amsmidinsert\noindent}

%%}}}

\endinput

%%{{{ Documentation / Notes 

%%{{{ ref notes and citations
## ref/lab notes

\cite[euclid]
\cite[euclid: Book III]

%%}}}

%%{{{ How to use mathrefs 
## How to use mathrefs

use inside any of:
\math
\longmath
\mathcols N
\compute
\stepproof
\context
etc.

Shortcuts:
\atag = \autotag
\mtag = \manutag
\dtag = \decotag
\itag = \aitag

Examples:

\math
1 + 1 = 2   \manutag[joke=Fun]  % Tags this as (Fun) and labels it 'joke'
\endmath

\compute
z+w  &= 2x  \by {by Lemma}  \autotag[foo] \\ % Tags this as (1) and labels it 'foo'
     &= x+x \by {by def.}   \autotag[bar] \\ % Tags this as (2) and labels it 'bar'
\endcompute

To decorate an existing tag use \decotag:

\math
1 + 1 = 3   \decotag[terror=joke with _*] % Tags this as (Fun*) and labels it 'terror'
\endmath

You may decorate autotagged stuff as well:

\math
z+w = 3x \decotag[altfoo=foo with *_']   % Tags this as (*1') and labels it 'altfoo'
\endmath

You may abuse decotag to have subequations.
\autoinvtag helps in case you don't need a (3) but just (3a), (3b):
(\manuinvtag works in a similar way)

\autoinvtag[xyzzy] % increments the eqnumber to 3 here but doesn't tag anything

Then use \decotag:

... \decotag[xyzzya=xyzzy with _a] \\
... \decotag[xyzzyb=xyzzy with _b] \\

Makes no sense to use a \subeqnumber counter for this,
since usually they are few and more easily managed like this.
%%}}}

%%{{{ How to use enums 
## How to use enums

\enumenter \enumleave are "low-level" commands
Low level configuration via:
\setenumlabel#Label
\setenumshowtaga#Atag
\setenumshowtagz#Ztag
\setenumntype#Level#Type
\setenumnsep#Level#Sep
You should only use \enum..\endenum

Tag types:
1: 1, 2, 3, ...
a: a, b, c, ...
i: i, ii, iii, ...
g: α, β, γ, ...
A: a, b, c, ...
I: i, ii, iii, ...
G: α, β, γ, ...
e: empty
.: bullets
o: circ
-: dashes
*: stars
u: user-defined

Note that some conf-tags overwrite each other (e.g. \dresstags, \li, ...)

### \tlist #DefaultTag:#Label .. \endtlist
Configuration:
    \spaciouslist
    \compactlist
    \dresstags with ##1_##2
    \nakedtags
    \withtag ##1
    \notags
    \eachitem ##1
Use:
    \li #Tag:#Label

### \enum #Numeraltype:#Label .. \endenum
Configuration:
    \dresstags with ##1_##2
    \nakedtags
    \sepwith ##1
    \eachitem ##1
    \whereu is ##1 (may be used to set userthesymbol, normally when #Numeraltype is u)
Use:
    \nn  % increases the current counter ("next number")
    \ti  % prints the enitemtag (naked)
    \si  % shows the enitemtag (pretty)
    \TI  % prints the enitemfulltag (naked)
    \SI  % shows the enimefulltag (pretty)
    \ei  % \nn\si  This is the most common use: \ei~nice stuff; \ei~more stuff; ...
    \EI  % \nn\SI

### \elist .. endelist
Use:
    \li #Comment:#Label  % \li from \tlist has been shadowed
    \LI #Comment:#Label  % \li from \tlist has been shadowed

Use \elist .. \endelist iff you want all tags of this list to be created via the enum

Examples:

\elist i:poutsidio
\eachitem hint
\li: first
\li: third
\endelist
Temos varios motivos:
\enum:i
\ei~pessoais;
\ei~profissionais;
\ei~doidos;
\endenum
e isso não nos preocupa não.
No outro lado olha nisso:
\elist g:fasoules
\eachitem example
\li:inter fase intermediaria;
\li:      fase inicial;
\endelist
Sacou?

Then check \refcount* and friends to ref.

%%}}}

%%{{{ How to humanly say n nouns 

## counting nouns
\humancount 2[m] homens
\humancount 2[f] mulheres
\humancountword 84[theorem]

%%}}}

%%}}}

